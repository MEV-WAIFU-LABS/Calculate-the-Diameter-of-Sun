\section{Conclusion}
% recapitulation of results and what we learned


Performing two complementary analysis: (i) the Taylor expansion of the visibility function, and (ii) the fitting of the visibility function, we have found  two values for the angular diameter of the sun, respectively,  $\Phi^{Tay}_{sun} = 37.43' \pm 5.33' $ and $\Phi^{Fit}_{sun}= 32.77' \pm 5.28'$. These results are compatibles between themselves and to the value registered in the literature,  $\Phi^{Actual}_{sun}= 31.1' \pm 0.6'$ \cite{wiki}. 

From the present results we  conclude that despite the simple character of the radio interferometer utilized in this experiment, it shown to be qualified to perform radio interference at radio wavelengths.


%%%%%%%%%%%%%%%%%%%%%%%%%%%%%%%%%%%%%%%%%%%%%%%%%%%%%%

\begin{thebibliography}{2}

\bibitem{michel} (1) Albert A. Michelson, Francis G. Pease, {\it Measurement of the diameter of alpha Orionis with the interferometer}, Astrophys. J. 53, 249-259 (1921).
\bibitem {sbu} (2) J. Koda \& J. Barret, {\it Stony Brook Radio Interferometer}, 2012.
\bibitem {wiki} (3) \url{http://www.wikipedia.org/} as 03/01/2012.
\bibitem {brand} (4) H. Bradt, {\it Astronomy Methods}, Cambridge University Press, 2007.
\bibitem{jackson} (5) J. D. Jackson, {\it Classical Electrodynamics}, (Third Edition), 1998. 


\bibitem[Dishpointer (2012)] {satbb} (6) Satellite Finder, \url{http://www.dishpointer.com/}, February/2012.
\bibitem[Navy (2012)] {sunbb} (7) Sun Altitude/Azimuth Table, \url{http://aa.usno.navy.mil/data/docs/AltAz.php}, February/2012.
\bibitem{loggerpro} (8)\url{ http://www.vernier.com/products/software/lp}.

\bibitem{math} (10) \url{ http://www.wolfram.com/mathematica/}.
\bibitem{root} (9) \url{http://root.cern.ch/drupal/}.
\end{thebibliography}

\clearpage

